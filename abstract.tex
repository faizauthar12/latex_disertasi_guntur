\newpage %Abstract
\addcontentsline{toc}{chapter}{ABSTRACT}
\begin{center}
%\begin{large}\textbf{Detection and Classification of Lower Molar Caries Using Dental Periapical Radiographic Image with Region Growing Segmentation  }\end{large}\\
 
 

\vspace{10mm}
\begin{large}\textbf{ABSTRACT}\end{large}
\end{center}
\vspace{5mm}

\begin{singlespace}
The application of information and communication technology is needed, especially in agriculture. Palm oil is the largest agricultural product in Indonesia and palm oil production is very important for the economy. In producing good oil palm production, good oil palm plantation management is needed. The main problems in this management, namely large-scale areas (oil palm plantation companies have a minimum area of 6,000 hectares that must be managed), plantation areas in remote areas, limited infrastructure access, and precision fertilization have difficulty obtaining accurate data based on the number of standing oil palm trees in a land area. This makes detecting and counting oil palm trees necessary. So far, traditional counting has been based on early records of oil palm tree planting or theoretical counting based on the spacing between oil palm trees within a hectare or block. These traditional methods are slow and inaccurate, and the status of damaged or dead oil palm trees is unknown. The use of technology is needed to be able to automatically and real-time monitor palm oil tree data and estimate its productivity. This use is needed in object detection in the form of images.

The use of images for object detection is needed in data preparation, such as analyzing, annotating or classifying the object. The method of annotating datasets has been done manually, one by one by providing boundary boxes. This data preparation consumes more than 70\% of the time in the deep learning lifecycle to become a dataset that can be used as training, validation, and testing data. This is the challenge for stakeholders.

This research aims to produce the development of object-based image analysis (OBIA) method to create datasets automatically by labeling a class on image data. The developed method uses a template matching classification algorithm. This algorithm as an initial template of palm tree image that has a key threshold value in determining the class of objects in the image. The BIRCH algorithm is used to reduce objects that are not detected into the palm tree class. The results of the training performance evaluation show that the model with the YOLOv7 algorithm is better with a best MAP accuracy of 0.993 and on testing of 0.997. Based on processing time, Gunadarma University's DGX-A-100 is better, at 2948 seconds of training compared to Google Colab Pro at 4847 seconds.

This research produced a prototype system that uses the algorithm model from YOLOv7 to be able to detect and count oil palm trees in a certain area from satellite images integrated with the Google Maps API. Based on the results of testing 4 blocks in the IPB-Cargil Oil Palm Education and Education Plantation that the presentation results were successfully detected by 97.67\%, and it is known that each detected oil palm tree is known to the location of the coordinate point to be able to monitor, manage, and estimate the productivity of oil palm trees.
\end{singlespace}

\noindent \\

\noindent Key words: OBIA, Deep Learning, YOLOv7, Oil Palm.