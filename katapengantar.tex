\newpage %Acknowledgment
\addcontentsline{toc}{chapter}{KATA PENGANTAR}
\begin{center}
\begin{large}\textbf{KATA PENGANTAR}\\\end{large}
\end{center}
\vspace{5mm}
 

\textit{Bismillahhirrohmaanirrohim}

\textit{Assalamu'alaikum Warahmatullahi Wabarakatuh} 

\textit{Alhamdulillah robil'aalamin} 

Segala Puji, Kebesaran, Kemulian, dan apa yang dilangit dan di bumi milik Allah SWT. Puji Syukur saya panjatkan kehadirat Allah SWT atas segala rahmat serta nikmat-Nya yang telah memberikan kemudahan serta kelancaran kepada Saya dalam penyelesaian Disertasi yang berjudul "Pengembangan Metode Deteksi Dan Menghitung Jumlah Pohon Kelapa Sawit Dari Sentinel 2 Imagery Menggunakan Metode \textit{OBJECT-BASED IMAGE ANALYSIS} (OBIA)". Disertasi ini merupakan syarat untuk memperoleh gelar Doktor dalam bidang Teknologi Informasi pada Program Doktor Teknologi Informasi, Program Pascasarjana, Universitas Gunadarma, dimana penulis telah menyelesaikan seluruh rangkaian proses studi Program Doktor sejak tahun 2019. Sepanjang proses penyusunan Disertasi ini, banyak pihak yang turut membantu baik secara moril maupun materil kepada saya. Untuk itu dengan segala kerendahan dan ketulusan hati, perkenankan saya mengucapkan terima kasih kepada:


\begin{enumerate}
\item Ibu Profesor Doktor E. S. Margianti, S.E., MM., selaku Rektor Universitas Gunadarma.
\item Bapak Profesor Doktor Insinyur Bambang Suryawan, MT., selaku Koordinator Program Pascasarjana Universitas Gunadarma.
\item Bapak Profesor Insinyur Busono Soerowirdjo, MSc., Ph.D., selaku Direktur Program Doktor Universitas Gunadarma.
\item Profesor Doktor Sarifuddin Madenda, selaku Ketua Program Doktor Teknologi Informasi Universitas Gunadarma sekaligus Ko-Promotor diselasela kesibukannya dengan sabar membimbing, mengarahkan, memberi masukan dan memotivasi dalam menyelesaikan disertasi
\item Bapak Profesor Doktor Eri Prasetyo Wibowo, selaku Sekretaris Program Doktor Teknologi Informasi Universitas Gunadarma.
\item Bapak Profesor Doktor Insinyur Kudang Boro Seminar, M.Sc. selaku Promotor yang dengan sabar memberikan membimbing, memotivasi, melakukan koreksi, memberi masukkan dan saran dalam menyelesaikan disertasi ini.
\item Bapak Profesor Doktor Insinyur Sudrajat, M.Sc., selaku Penguji Luar terima kasih atas waktu, kesediaan, dan masukkan, serta saran disertasinya bagi saya.
\item Bapak Profesor Suryadi Harmanto, SSi., MMSI, Bapak Doktor rer. nat. I Made Wiryana dan Ibu Doktor Detty Purnamasari selaku penguji dalam yang telah memberi banyak masukan dan saran perbaikan, sehingga disertasi ini semakin berkualitas.
\item Bapak Doktor Irwan Bastian yang telah memberikan dukungan kepada saya dapat melaksanakan kuliah Program Doktor Teknologi Informasi di Universitas Gunadarma.
\item Ayahanda tercinta Baikusnendro dan Ibunda tercinta Mennik Trihastuti, yang selalu memberikan doa yang terbaik dan motivasi, serta kedua adik saya Finsa Dwi Hestu Fikriansya, dan Hilmi Hestu Saputra yang selalu mendukung secara moril.
\item Rekan-rekan angkatan 25 Program Doktor Teknologi Informasi Universitas
Gunadarma yang selalu memberikan semangat, dan Mas Bonang Waspadadi Ligar, serta Sumaiyah Fitriandini yang berjuang bersama, serta diskusi.
\item Ibu-ibu di Sekretariat Program Doktor Teknologi Informasi Universitas Gunadarma Ibu Doktor Diny Wahyuni, Ibu Doktor Reni Diah Kusumawati, Ibu Doktor Aini Suri Talita, dan Ibu Doktor Dety Purnamasari yang sangat membantu dalam administrasi penyelesaian disertasi ini.
\item Teman-teman Bidang Kemahasiswaan Universitas Gunadarma, komunitas GDSC UG, Gunadarma IO, UGTV, Tim Teknis UG dan sahabat saya Muhammad Rifqi Al Furqon, Manfred Michael, Evan Sakti Endi, dan Muhammad Alfaiz Khisma Authar yang telah membantu dalam penyelesaian disertasi ini.
\end{enumerate}

Semoga Allah SWT memberikan limpahan kebaikan dan pahala atas semua perhatian dan dukungan yang Bapak, Ibu, Saudara sekalian berikan kepada saya. Harapan saya agar Disertasi ini memberikan manfaat nyata bagi semua pihak yang berkepentingan. Saya mengharapkan kritik dan saran untuk perbaikan pada masa yang akan datang.


\begin{flushleft}
\textit{Wassalamu'alaikum Warahmatullahi Wabarakatuh}
\end{flushleft}

\begin{flushleft}
Jakarta, 10 Mei 2023\\
%Penulis
\vspace{2.5 cm}
(Guntur Eka Saputra)

\end{flushleft} 
