\chapter{PENUTUP}
\section{Kesimpulan}
\hspace{1,2cm}
Berdasarkan permasalahan dan hasil penelitian yang telah dilakukan, maka dapat ditarik kesimpulan secara umum bahwa purwarupa sistem berbasis web menggunakan model OBIA dan \textit{Deep Learnig} (DL) yang telah dihasilkan dari penelitian ini mampu mendeteksi dan menghitung tegakkan kelapa sawit dalam satu wilayah tertentu secara cepat, akurat, dan \textit{up-to-date}. Secara khusus penelitian ini dapat disimpulkan sebagai berikut:

\begin{enumerate}
	\item Algoritma \textit{template} matching dan \textit{Balanced Iterative Reducing and Clustering Using Hierarchies} (BIRCH) dengan pendekatan OBIA berhasil digunakan untuk dapat membuat dataset otomatis dengan diberikan anotasi atau pelabelan 'oil palm' pada 205 citra dataset primer dengan waktu yang dibutuhkan untuk anotasi dataset otomatis selama 30 menit 24 detik. Berdasarkan ujicoba dengan 10 data citra dihasilkan 89,90\% berhasil dideteksi sebagai dataset 'oil palm'.
	
	\item Berdasarkan hasil evaluasi performance yang digunakan dari YOLOv5, YOLOv6, dan YOLOv7 dengan menggunakan Google Colab Pro dan DGX-A-100 Universitas Gunadarma bahwa model dengan algoritma YOLOv7 pada pelatihan lebih baik dibandingkan YOLOv5 dan YOLOv6 dengan hasil \textit{Precision}, \textit{Recall}, \textit{F1-Score}, dan \textit{best mAP @0.5}. pada pelatihan secara berurutan adalah 0,970, 0,976, 0,973, dan 0,993, sedangkan pada hasil pengujian mengalami peningkatan yaitu 0.984, 0.989, 0.986, dan 0.997. Hal ini menunjukkan bahwa model dapat menemukan dan mempelajari pola dari dataset dengan baik, sehingga dapat mengenali pola pada data pengujian, dan pendugaan ketelitian tegakkan pohon kelapa sawit dengan akurasi best mAP 0,997 atau 97,70\% lebih baik dibandingkan dengan penelitian sebelumnya. Berdasarkan segi waktu komputasi yang dibutuhkan untuk pelatihan dan pengujian, DGX-A-100 Universitas Gunadarma lebih baik dibanding Google Colab Pro, tercatat waktu pelatihan pada YOLOv7 sebesar 2948 detik, lebih baik dibandingkan dengan Google Colab Pro yaitu 4847 detik, sedangkan pada proses pengujian lebih besar pada Google Colab Pro sebesar 53 detik, berbeda 1 detik dengan DGX-A-100 Universitas Gunadarma. 
	
	\item Hasil purwarupa sistem berhasil dibuat dan diterapkan dengan model CNN dari algoritma YOLOv7 untuk mendeteksi dan menghitung pohon kelapa sawit, serta menghasilkan identifier berdasarkan titik koordinat latitude dan longitude untuk setiap pohon kelapa sawit berdasarkan citra satelit yang terintegrasi dengan Google Maps API, sehingga bisa menjadi identifier yang unik untuk memonitoring pohon kelapa sawit. Berdasarkan hasil uji coba pada 4 blok area Kebun Pendidikan dan Penelitian Kelapa Sawit IPB-Cargil, maka hasil presentase berhasil dideteksi objek kelapa sawit sebesar 97.67\%.
	
\end{enumerate}

\section{Saran}
\hspace{1,2cm}
Berdasarkan hasil yang dicapai dari hasil penelitian bahwa peneliti mengakui bahwa masih dalam lingkup keterbatasan dalam mengumpulkan dataset primer yang dapat digunakan sebagai pembelajaran oleh mesin, sehingga dapat menambah area dan cakupan dataset primer untuk dapat melatih model lebih baik. Data yang terwakili perkebunan kelapa sawit masih sedikit, dan dapat digunakan dengan penelitian dengan perkebunan kelapa sawit dengan tanah gambut. Penelitian ini dapat dikembangkan dan diterapkan pada model CNN lainnya, seperti YOLOv8 atau \textit{Region Based Convolutional Neural Network} (R-CNN) untuk mendeteksi objek dan mendukung manajemen perkebunan, seperti pemantauan perkebunan secara otomatis, mendeteksi dan menghitung pohon kelapa sawit, nutrisi pohon, pemetaan, dan pemupukan. Dapat meningkatkan performa model untuk dapat membuat waktu komputasi pada pelatihan dan pengujian lebih cepat. Pengembangan purwarupa sistem sudah berhasil mendeteksi, menghitung, dan mendapatkan letak atau titik koordinat dari setiap objek kelapa sawit yang berhasil dideteksi, sehingga selanjutnya, dapat menjadi acuan dan membantu para \textit{stakeholder} untuk pemantauan dan pengambilan keputusan. 